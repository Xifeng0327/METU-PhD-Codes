
% Homework template for MA 614, Spring 2011.  When a line begins with the percent sign, the typesetter ignores it.  So, use percent signs at the beginning of lines to insert comments to yourself.


% Set the document class.  The command [11pt] sets the font at 11 point, which is nicer to read.  The default would be 10pt
\documentclass[11pt]{amsart} 


% Call packages that allow you to invoke certain mathematical symbols.
\usepackage{amssymb,amsmath,amsthm}
\usepackage[framed,numbered,autolinebreaks,useliterate]{mcode}

% Set the title, author, and date information.
\title{Homework 3}
\author{Anil Aksu}
\date{\today}


% Formally begin the document and make the title.
\begin{document}
\maketitle

\section*{Problem 1 }
Show that for an electric field of the form: 
\begin{equation*}
\vec{E}(\vec{x},\tau,t)=\vec{E_0}(\vec{x},\tau)\cos(\omega t- \vec{k}\vec{x}).
\end{equation*}
the magnetic field is given by
\begin{equation*}
\vec{B}(\vec{x},\tau,t)=-\frac{1}{\omega}\left \{  \left [\vec{\nabla}\times \vec{E_0}(\vec{x},\tau) \right ]\sin(\omega t -\vec{k}\vec{x})-\left [\vec{k}\times  \vec{E_0}(\vec{x},\tau) \right ]\cos(\omega t- \vec{k}\vec{x})\right \}.
\end{equation*}
\\
\textbf{Solution:}\\
by Maxwell's equation the relation between the magnetic field $\vec{B}$ and the electric field $\vec{E}$ can be given as:
\begin{equation}
\label{eq:1}
\frac{\partial \vec{B}}{\partial t}=-\vec{\nabla}\times\vec{E}.
\end{equation}
Therefore, if we replace the electric field given in problem to equation \ref{eq:1}, the rate of change of the magnetic field can be obtained as:
\begin{equation}
\label{eq:2}
\frac{\partial \vec{B}}{\partial t}= -\left [\vec{\nabla}\times \vec{E_0}(\vec{x},\tau) \right ]\cos(\omega t -\vec{k}\vec{x})-\left [\vec{k}\times  \vec{E_0}(\vec{x},\tau) \right ]\sin(\omega t- \vec{k}\vec{x}).
\end{equation}
Note that here $\tau$ is assumed to be slowly varying time, therefore during the time integration over one wave period $2\pi / \omega$, it is assumed to be constant, therefore the magnetic field $\vec{B}$ can be given as:
\begin{equation}
\label{eq:3}
\vec{B}=\int \frac{\partial \vec{B}}{\partial t} \mathrm{d} t=-\frac{1}{\omega}\left \{  \left [\vec{\nabla}\times \vec{E_0}(\vec{x},\tau) \right ]\sin(\omega t -\vec{k}\vec{x})-\left [\vec{k}\times  \vec{E_0}(\vec{x},\tau) \right ]\cos(\omega t- \vec{k}\vec{x})\right \}.
\end{equation}
\newpage
 \section*{Problem 2 }
An electron of charge $q_e=-e$ and mass $m_e$ and an proton of charge $q_e=e$ and mass $m_i=m_p$ are initially at rest at $\mathbf{x}=(0,0,0)$ in a magnetic field $\vec{B}=B_0 \hat{z}$. An electric field is then turned on at $t=0$ and increased linearly until $t_1=\frac{20\pi m_i}{e B_0}$ at which point electric field held constant,
\begin{equation*}
\vec{E}=
\left\{\begin{matrix}
0 & t<0\\ 
E_0(t/t_1)\hat{y} &0\leq t\leq t_1 \\ 
E_0 \hat{y} & t\geq t_1
\end{matrix}\right.
\end{equation*}
Find the total current as a function of time due to drifts of two particles (neglect
the current due to the fast Larmor oscillation).
\\
\textbf{Solution:}\\
To derive any arbitrary drift motion due to the presence of a magnetic field $\mathbf{B}$, let's write the equation of motion:
\begin{equation}
\label{eq:4}
m\dot{\mathbf{v}}=q(\mathbf{E}+\mathbf{v}\times \mathbf{B}).
\end{equation}
Under these condition,
Since we have magnetic field only in $\hat{z}$ direction, it only causes an acceleration in $\hat{x}$ and $\hat{y}$ directions, therefore the equations of motion can be given as:
\begin{equation}
\label{eq:5}
\dot{v}_x=qv_yB_0/m,
\end{equation}
\begin{equation}
\label{eq:6}
\dot{v}_y=q(E/m-v_xB_0/m).
\end{equation}
Equation \ref{eq:5} and equation \ref{eq:6} can be combined by taking derivative of equation \ref{eq:6} and replacing equation \ref{eq:6} into it. The resultant equation can be obtained as:
\begin{equation}
\label{eq:7}
\ddot{v}_{x}+(\frac{q B_{0}}{m})^2 v_{x}=\frac{q^2 B_{0}E}{m^2}.
\end{equation}
Therefore, the solution can be given as a combination of homogeneous and particular solution:
\begin{equation}
\label{eq:8}
v_{x}=v_{x}^h+v_{x}^p.
\end{equation}
The homogeneous part is a combination of cosine and sine function with a frequency $\omega=q B_0/m$ and the non-homogeneous part is a polynomial in form of $a t+ b$, therefore the solution can be explicitly given as:
\begin{equation}
\label{eq:9}
v_{x}=c_1 \cos{\omega t}+ c_2 \sin{\omega t}+ a t+b.
\end{equation}
for $0<t \leq t_1$, the particular solution can be given as:
\begin{equation}
\label{eq:10}
v_{x}^p=\frac{E_0 }{B_0 t_1}t.
\end{equation}
If we replace it into equation \ref{eq:5}, the velocity in $y$ direction can be found as:
\begin{equation}
\label{eq:11}
v_{y}^p=\frac{E_0 }{B_0 t_1 \omega}.
\end{equation}
After imposing initial conditions, it comes out that the total velocity field can be given as:
\begin{equation}
\label{eq:12}
\begin{bmatrix}
v_x\\ 
v_y
\end{bmatrix}
=
\begin{bmatrix}
\frac{E_0 }{B_0 t_1}t-\frac{E_0 }{B_0 t_1 \omega}\sin{\omega t}\\ 
\frac{E_0 }{B_0 t_1 \omega}(1-\cos{\omega t})
\end{bmatrix}.
\end{equation}
At $t=t_1$, the velocity field can be given as:
\begin{equation}
\label{eq:13}
\begin{bmatrix}
v_x\\ 
v_y
\end{bmatrix}
=
\begin{bmatrix}
\frac{E_0 }{B_0}-\frac{E_0 }{B_0 t_1 \omega}\sin{\omega t_1}\\ 
\frac{E_0 }{B_0 t_1 \omega}(1-\cos{\omega t_1})
\end{bmatrix}.
\end{equation}
This can be directly used as an initial condition for $t>t_1$. Similar to the previous time interval, the total solution for $t>t_1$ can be given as:
\begin{equation}
\label{eq:14}
\begin{bmatrix}
v_x\\ 
v_y
\end{bmatrix}
=
\begin{bmatrix}
c_1 \cos{\omega t} + c_2 \sin{\omega t} +\frac{E_0}{B_0} \\ 
-c_1 \sin{\omega t} + c_2 \cos{\omega t}
\end{bmatrix}.
\end{equation}
The velocities have to match at $t=t_1$, therefore:
\begin{equation}
\label{eq:15}
\begin{bmatrix}
 \cos{\omega t_1} & \sin{\omega t_1} \\ 
- \sin{\omega t_1} & \cos{\omega t_1}
\end{bmatrix}
\begin{bmatrix}
c_1  \\ 
 c_2
\end{bmatrix}
+
\begin{bmatrix}
\frac{E_0 }{B_0}\\ 
0
\end{bmatrix}
=
\begin{bmatrix}
-\frac{E_0 }{B_0 t_1 \omega}\sin{\omega t_1}\\ 
\frac{E_0 }{B_0 t_1 \omega}(1-\cos{\omega t_1})
\end{bmatrix}
+
\begin{bmatrix}
\frac{E_0 }{B_0}\\ 
0
\end{bmatrix}
.
\end{equation}
After cancelling $E_0/B_0$ at both sides and inverting the matrix and multiplying it with the right hand side of the equation \ref{eq:15}, constants $c_1$ and $c_2$ can be found as:
\begin{equation}
\label{eq:16}
\begin{bmatrix}
 \cos{\omega t_1} & -\sin{\omega t_1} \\ 
 \sin{\omega t_1} & \cos{\omega t_1}
\end{bmatrix}
\begin{bmatrix}
-\frac{E_0 }{B_0 t_1 \omega}\sin{\omega t_1}\\ 
\frac{E_0 }{B_0 t_1 \omega}(1-\cos{\omega t_1})
\end{bmatrix}
=
\begin{bmatrix}
c_1  \\ 
 c_2
\end{bmatrix}
=
\begin{bmatrix}
-\frac{E_0 }{B_0 t_1 \omega}\sin{\omega t_1} \\ 
 \frac{E_0 }{B_0 t_1 \omega}(\cos{\omega t_1}-1)
\end{bmatrix}
.
\end{equation}
As a result, the velocity field after $t>t_1$ is obtained as:
\begin{equation}
\label{eq:17}
\begin{bmatrix}
v_x\\ 
v_y
\end{bmatrix}
=
\begin{bmatrix}
-\frac{E_0 }{B_0 t_1 \omega}\sin{\omega t_1} \cos{\omega t} +  \frac{E_0 }{B_0 t_1 \omega}(\cos{\omega t_1}-1) \sin{\omega t} +\frac{E_0}{B_0} \\ 
\frac{E_0 }{B_0 t_1 \omega}\sin{\omega t_1} \sin{\omega t} +  \frac{E_0 }{B_0 t_1 \omega}(\cos{\omega t_1}-1)\cos{\omega t}
\end{bmatrix}.
\end{equation}
The velocity field came out to be independent of the sign of the charge. The current is given as:
\begin{equation}
\label{eq:18}
\mathbf{j}=n_{e} e (\vec{u}_e-\vec{u}_i).
\end{equation}
Under this formulation, the only parameter different is the frequency due to the differences in mass which are $m_e$ and $m_i$. Let's denote corresponding frequencies by $\omega_e$ and $\omega_i$. Therefore, for $t<t_1$ the current can be given as:
\begin{equation}
\label{eq:19}
\begin{bmatrix}
j_x\\ 
j_y
\end{bmatrix}
=
n_{e}e
\begin{bmatrix}
\frac{E_0 }{B_0 t_1 }(\frac{\sin{\omega_i t}}{\omega_i}-\frac{\sin{\omega_e t}}{\omega_e})\\ 
\frac{E_0 }{B_0 t_1 }(\frac{(1-\cos{\omega_e t})}{\omega_e}-\frac{(1-\cos{\omega_i t})}{\omega_i})
\end{bmatrix}.
\end{equation}
And also for $t>t_1$ the current can be given as:
\begin{equation}
\label{eq:20}
\begin{bmatrix}
j_x\\ 
j_y
\end{bmatrix}
=
n_{e}e
\begin{bmatrix}
\frac{E_0 }{B_0 t_1 }(\frac{\sin{\omega_i t_1} \cos{\omega_i t}}{\omega_i}-\frac{\sin{\omega_e t_1}\cos{\omega_e t}}{\omega_e})-\frac{E_0 }{B_0 t_1 }(\frac{\sin{\omega_i t} (\cos{\omega_i t_1}-1)}{\omega_i}-\frac{\sin{\omega_e t}(\cos{\omega_e t_1}-1)}{\omega_e})\\ 
-\frac{E_0 }{B_0 t_1 }(\frac{\sin{\omega_i t_1} \sin{\omega_i t}}{\omega_i}-\frac{\sin{\omega_e t_1}\sin{\omega_e t}}{\omega_e})-\frac{E_0 }{B_0 t_1 }(\frac{\cos{\omega_i t} (\cos{\omega_i t_1}-1)}{\omega_i}-\frac{\cos{\omega_e t}(\cos{\omega_e t_1}-1)}{\omega_e})
\end{bmatrix}.
\end{equation}
\end{document}
