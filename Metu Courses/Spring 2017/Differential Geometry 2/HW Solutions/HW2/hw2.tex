
% Homework template for MA 614, Spring 2011.  When a line begins with the percent sign, the typesetter ignores it.  So, use percent signs at the beginning of lines to insert comments to yourself.


% Set the document class.  The command [11pt] sets the font at 11 point, which is nicer to read.  The default would be 10pt
\documentclass[11pt]{amsart} 


% Call packages that allow you to invoke certain mathematical symbols.
\usepackage{amssymb,amsmath,amsthm}


% Set the title, author, and date information.
\title{Homework 2}
\author{Anil Aksu}
\date{\today}


% Formally begin the document and make the title.
\begin{document}
\maketitle

\section{Problem 1 }

Given a connection $\nabla$ with torsion $T$ and curvature $R$, prove that
\\
\subsection*{A}
\begin{equation*}
\wp_{X,Y,Z} \left ( R(X,Y,Z)+T(T(X,Y),Z)+\nabla_{Z}T(X,Y) \right ))=0.
\end{equation*}
\\
\textbf{Solution:}\\
Let's start the solution by defining torsion $T$ and curvature $R$. The torsion of $\nabla$ is the tensor field $T=T_{\nabla}$ of bidegree $(1,2)$ defined for each $X,Y,Z\in TM$ by 
\begin{equation*}
T(X,Y)=\nabla(X,Y)-\nabla(Y,X)-\left [ X,Y \right ].
\end{equation*} 
And also the curvature of $\nabla$ is defined as:
\begin{equation*}
R(X,Y,Z)=\nabla(X,\nabla(Y,Z))-\nabla(Y,\nabla(X,Z))-\nabla(\left [ X,Y \right ],Z).
\end{equation*} 
Let's write the relation between torsion and curvature explicitly,
\begin{equation}
\label{eq:1}
\begin{split}
R(X,Y,Z)+T(T(X,Y),Z)+\nabla_{Z}T(X,Y)=\nabla(X,\nabla(Y,Z))-\nabla(Y,\nabla(X,Z))
\\
-\nabla(\left [ X,Y \right ],Z)+\nabla(\nabla(X,Y),Z)-\nabla(Z,\nabla(X,Y))-\left [ \nabla(X,Y),Z \right ]
\\
+\nabla(Z,\nabla(X,Y))-\nabla(Z,\nabla(Y,X))-\nabla(Z,\left [ X,Y \right ]).
\end{split}
\end{equation}
The term $\nabla(Z,\nabla(X,Y))$ cancels out and it reduces to:
\begin{equation}
\label{eq:2}
\begin{split}
R(X,Y,Z)+T(T(X,Y),Z)+\nabla_{Z}T(X,Y)=\nabla(X,\nabla(Y,Z))-\nabla(Y,\nabla(X,Z))
\\
-\nabla(\left [ X,Y \right ],Z)+\nabla(\nabla(X,Y),Z)-\left [ \nabla(X,Y),Z \right ]
-\nabla(Z,\nabla(Y,X))-\nabla(Z,\left [ X,Y \right ]).
\end{split}
\end{equation}
Moreover, the next term in cycle with order $(Z,X,Y)$ can be given as:
\begin{equation}
\label{eq:3}
\begin{split}
R(Z,X,Y)+T(T(Z,X),Y)+\nabla_{Y}T(Z,X)=\nabla(Z,\nabla(X,Y))-\nabla(X,\nabla(Z,Y))
\\
-\nabla(\left [ Z,X \right ],Y)+\nabla(\nabla(Z,X),Y)-\left [ \nabla(Z,X),Y \right ]
-\nabla(Y,\nabla(X,Z))-\nabla(Y,\left [ Z,X \right ]).
\end{split}
\end{equation}
Finally, the term $(Y,Z,X)$ is given as:
\begin{equation}
\label{eq:4}
\begin{split}
R(Y,Z,X)+T(T(Y,Z),X)+\nabla_{X}T(Y,Z)=\nabla(Y,\nabla(Z,X))-\nabla(Z,\nabla(Y,X))
\\
-\nabla(\left [ Y,Z \right ],X)+\nabla(\nabla(Y,Z),X)-\left [ \nabla(Y,Z),X \right ]
-\nabla(X,\nabla(Z,Y))-\nabla(X,\left [ Y,Z \right ]).
\end{split}
\end{equation}
Note that
\begin{equation}
\label{eq:5}
\begin{split}
\nabla(\nabla(Y,Z),X)-\left [ \nabla(Y,Z),X \right ]
-\nabla(X,\nabla(Z,Y))=\nabla(X,\nabla(Y,Z))-\nabla(X,\nabla(Z,Y))
\\
=\nabla(X,\left[ Y,Z \right ]).
\end{split}
\end{equation}
After replacing this result into equation \ref{eq:4}, it reduces to:
\begin{equation}
\label{eq:6}
\begin{split}
R(Y,Z,X)+T(T(Y,Z),X)+\nabla_{X}T(Y,Z)=\nabla(Y,\nabla(Z,X))-\nabla(Z,\nabla(Y,X))
\\
-\nabla(\left [ Y,Z \right ],X).
\end{split}
\end{equation}
It also means that last four terms of equation \ref{eq:2} and \ref{eq:3} drops out and if they are all added together, the following  summation is obtained:
\begin{equation}
\label{eq:7}
\begin{split}
\nabla(Y,\nabla(Z,X))-\nabla(Z,\nabla(Y,X))-\nabla(\left [ Y,Z \right ],X)+\nabla(X,\nabla(Y,Z))-\nabla(Y,\nabla(X,Z))
\\
-\nabla(\left [ X,Y \right ],Z)+\nabla(Z,\nabla(X,Y))-\nabla(X,\nabla(Z,Y))-\nabla(\left [ Z,X \right ],Y)
\end{split}
\end{equation}
In summation \ref{eq:7}, 
\begin{equation}
\label{eq:8}
\nabla(Y,\nabla(Z,X))-\nabla(Y,\nabla(X,Z))-\nabla(\left [ Z,X \right ],Y)=\nabla(Y,\left [ Z,X \right ])-\nabla(\left [ Z,X \right ],Y)=\left [ Y,\left [ Z,X \right ] \right ],
\end{equation}
\begin{equation}
\label{eq:9}
\nabla(Z,\nabla(X,Y))-\nabla(Z,\nabla(Y,X))-\nabla(\left [ X,Y \right ],Z)=\nabla(Z,\left [ X,Y \right ])-\nabla(\left [ X,Y \right ],Z)=\left [ Z,\left [ X,Y \right ] \right ],
\end{equation}
\begin{equation}
\label{eq:10}
\nabla(X,\nabla(Y,Z))-\nabla(X,\nabla(Z,Y))-\nabla(\left [ Y,Z \right ],X)=\nabla(X,\left [ Y,Z \right ])-\nabla(\left [ Y,Z \right ],X)=\left [ X,\left [ Y,Z \right ] \right ].
\end{equation}

Therefore, the summation \ref{eq:7} can be written more compactly as:
\begin{equation}
\left [ Y,\left [ Z,X \right ] \right ]+\left [ Z,\left [ X,Y \right ] \right ]+\left [ X,\left [ Y,Z \right ] \right ]=0.
\end{equation}

\end{document}





